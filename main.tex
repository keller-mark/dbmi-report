\documentclass[12pt, letterpaper]{article}
\usepackage[margin=0.8in]{geometry}
\usepackage[utf8]{inputenc}
\usepackage[numbers]{natbib}
\usepackage{graphicx}
\usepackage{xcolor}
\usepackage{wrapfig}

\usepackage{hyperref}

\title{HMS Summer Institute in Biomedical Informatics}
\author{Mark Keller \thanks{advised by Professor Nils Gehlenborg}}
\date{Summer 2019}

\begin{document}
\maketitle

\begin{abstract}
Abstract here...
\end{abstract}

\section{Introduction}
\subsection{Background}
% Importance of single-cell analysis
Single-cell genomics and transcriptomics enable cell-level measurement of quantities such as gene expression, mutations, and methylation.
Single-cell data reveals cellular heterogeneity and captures the functional states of individual cells that would be hidden by measurements of bulk cell populations that only capture averages~\cite{shapiro2013single}.
Cell types can be teased apart in single-cell data using computational approaches such as dimensionality reduction and clustering based on gene expression profiles~\cite{stegle2015computational}.

% Need for visualization
Visualization of 2D embeddings of cells can facilitate identification of distinct cell populations that group into clusters or subclusters~\cite{wang2017visualization}.
However, such embeddings are dependent on the dimensionality reduction method employed and its parameters.
Often, multiple dimensionality reduction methods are compared, including PCA, t-SNE, and UMAP~\cite{becht2019dimensionality}.
Technologies that quantify gene expression may also preserve spatial information, for example RNA flourescence \textit{in situ} hybridization (FISH).
In some tissues, it is likely that cells located close to each other are of the same type~\cite{stegle2015computational}.

Vitessce is a web-based visualization tool for single-cell experiment data, including spatial data from multiple modalities and scatterplot data from arbitrary dimensionality reduction methods.
Visualizations in Vitessce are interactive, with customizable zoom levels, viewer sizes, tooltips, and color encodings.
To display spatial data, including cells and molecules, Vitessce leverages performant geospatial web technologies.


\subsection{Related work}
% Existing visualizations tools and browsers
There are currently several tools to visualize spatial single cell transcriptomics data. 
Web-based tools include the UCSC Cell Browser \cite{nowakowski2017spatiotemporal}.

% Existing "static" visualizations in publications

% Data generated by HuBMAP
Following the model of the human genome project which focused resources across institutions to discover a consensus human genome sequence, projects such as ENCODE~\cite{encode2004encode}, GTEx~\cite{lonsdale2013genotype}, the Human Cell Atlas~\cite{regev2017science}, and the 4D Nucleome~\cite{dekker20174d} have been envisioned and executed.
The NIH-sponsored Human Biomolecular Atlas Program (HuBMAP) is using this model to map the human body at the single cell level in a limited number of subjects.
HuBMAP aims to develop spatial mappings of cells and molecules, with new coordinate frameworks that allow querying across levels, from organ to tissue to cell to molecule.
Learning from the struggles of past biomedical and genomic data collection efforts that have developed data portals and visualization tools as afterthoughts, HuBMAP was conceived with the HuBMAP Integration, Visualization, and Engagement (HIVE) group responsible for articulating data access needs from the start~\cite{snyder2019mapping}.

% HuBMAP conference experience

% Current questions



\subsection{Contributions}
% Introduction to Vitessce
Vitessce...

\section{Methods}

% Cartographic technologies
Vitessce performs spatial visualization by leveraging libraries developed with a focus on cartography and mapping.
The open source Uber-maintained Deck.gl JavaScript library is powerful because it implements reactive updates that are similar to those used by React.
Deck.gl performs diffing on the data passed to each view layer to detect updates and invalidate the current state, just as React performs diffing on the virtual DOM in response to data changes to determine when to update the browser's DOM.

Often, it makes sense to render contextual elements in the DOM rather than in a visualization layer, whether for performance reasons or to keep the visualization clean to in preparation for downloading.
To synchronize HTML element positions with positions of data points in visualization layers, we perform projections from the data coordinate space to the browser coordinate space.

% Linked cell hover

% Tooltips

% Component re-use
The development of Vitessce has focused on separating logic into components that can operate independently to enable them to be imported and re-used by other projects, including HuBMAP data portals and tissue viewers.
To achieve this goal, Vitessce does not maintain a global state that is used by child components.
Instead, there is an event-based mechanism, with wrapper subscriber components that pass state down to children along with publisher update functions.
This allows external usage of components to be done by setting up custom publishing and subscription wrappers rather than using the Vitessce-specific wrappers.
A drawback of this approach is that events are asynchronous, so implementing a history mechanism may prove to be difficult, as there is no one source of truth for events.

% Cell selection, rectangle and polygon

% Cell set management (and set management in general, with independent utility functions)

% Cell set hierarchy

% Set operations


% Tech debt
% - React hooks
% - CSS -> SCSS
% - 


\section{Conclusions}
% Brief summary
In this paper...

% Future directions
% - 
Next steps...



\bibliography{main}{}
\bibliographystyle{plainnat}

\end{document}
